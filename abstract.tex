%%%% Time-stamp: <2013-02-25 10:31:01 vk>


\chapter*{Abstract}
\label{cha:abstract}

Borrowing a concept from statistical physics, namely the theory of Brownian motion or the apparently random movement of a particle in a fluid, this work aims to describe the behaviour of young honeybees at the level of individual thermokinesis or the driven movement in a temperature gradient.
The main equation, to which the bees trajectories are compared, and with which they are analyzed, is the Langevin equation considering the movement as a combination of propulsion, dissipation, drift and noise. \\
Using well understood methodology like the mean squared displacement of diffusive particles and the power spectral density of a given time series, this approach allows to qualitatively and partially quantitatively distinguish between different types of behaviour and yields a set of constants when compared to the tracking data of 136 experiments with single bees in a confined and inhomogeneously temperated arena. \\
Although young honeybees in general prefer warmer temperatures when exposed to a gradient, the analysis shows, that they are either unable or unwilling to reliably walk in the direction of greatest ascent in the presented experimental setting. They do however show higher probabilities of stopping and longer durations of staying at temperatures, that are convenient enough. \\





%\glsresetall %% all glossary entries should be used in long form (again)
%% vim:foldmethod=expr
%% vim:fde=getline(v\:lnum)=~'^%%%%\ .\\+'?'>1'\:'='
%%% Local Variables:
%%% mode: latex
%%% mode: auto-fill
%%% mode: flyspell
%%% eval: (ispell-change-dictionary "en_US")
%%% TeX-master: "main"
%%% End:
