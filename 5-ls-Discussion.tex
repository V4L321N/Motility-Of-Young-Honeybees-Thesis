\chapter{Discussion}
\label{cha:Discussion}

%Of the methods that are described in chapter \ref{cha:Methods}, some are offering a better quantification of the ``biological system bee'' and some are due to their 

Of all the methods that deliver results, some perform quite well and some do not. Admittedly this may seem like a generic statement and intrinsically true, but it is not that trivial:
\\
\\
Comparing individual bees, an already highly complex system expressing itself in a manifold of different behaviours, means dealing with a considerable amount of noise. Not only within the data itself but as well on the macroscopic layer above, where the different behaviours are described and defined, and then another one at the level of human observation. Furthermore, the ability to distinct the types in a quantitative and ascertained manner is highly dependent on the bees actions, where unknown underlying factors are not (yet) accounted for. The boundaries between the types are blurred, and a random walking bee may by chance enter the optimum and switch to immobile behaviour, which in a slightly diverging configuration would be described as an uphill walk in the temperature gradient.
So in the end every analysis looks at the system from a different perspective, and in this context every bit of insight can be helpful.
Actually every method used, allows to at least draw qualitative conclusions, some of them raise new questions and perhaps some of them need to be improved.
\\
\\
The data itself is sufficient to describe the movement with the presented attempt but brings troubles with its volatile definition of the optimum and pessimum as the temperature uncertainties (especially in the pessimum) do not allow to clearly divide the experiments and look at the influence of the gradient. However, it seems that the absolute difference in temperature between the optimum and the pessimum has little to no influence on most of the bees. The ones that are susceptible and turn predominantly towards the goal are eventually again showing all kinds of behaviour. Figure \ref{fig:D_vs_a_theta} reveals that even the 16 carefully picked paragons spread randomly, with an $a_{\theta}$ (the cutoff frequency and indicator for gradient dependence) rarely higher than $0.005 \; 1/s$ and a $D_{\theta}$ (the radial diffusion coefficient) which spans over four magnitudes from below $10^{-1} \; cm^{2}/s$ to above $10^{2} \; cm^{2}/s$. The power spectrum $S_{\theta}$ is in most cases following a slope of $1/\omega^{2}$ again indicating randomness in the turning behaviour. Nevertheless, the examples that have a (by a magnitude) higher cutoff frequency of around $a_{\theta} \approx 0.02 \; 1/s$ (i.e.: $GF_{2}$, $WF_{2}$, $RW_{3}$), surely are ones that actually did end up in the optimum (already hinted in figure \ref{fig:Well_Beehaved}) although in experiments $WF_{2}$ and $RW_{3}$ manually being classified differently. In the end this means that of the angular Langevin equation (eq. \ref{eq:Langevin_theta}) only the noise $\eta_{\theta}$ multiplied by an constant factor $\sqrt{2D_{\theta}}$ remains. 
\\
\\
It is with the velocity that we see the first real differences and a pointer towards what really is going on. The histograms in figure \ref{fig:Hist_velocity} reveal that there are two values that dominate the act of walking itself by being the two modes of a bimodal distributed probability density. The second mode, through which all velocities beyond $0.5 cm/s$ are represented, can be described by a Normal distribution, and their means show higher values for less directional behaviour, implicating that a uphill walking bee is slower and more cautious in its steps.
Keeping in mind not to over-interpret the underlying mechanisms, this too could be dependent on several unconsidered and underestimated circumstances: Firstly, the bees had free access to honey before doing the experiments, however they have not been actively fed, which would reduce possibly effects caused by low energy levels, such as getting slower or stopping. Secondly, when looking at table \ref{tab:AllData} and the two blocks of experiments BT08A-1 through 4 and BT08B-1 through 4 the velocities seem to be similar within the block and changing by a factor of 2 in between them. This is observable throughout the results and raises the question, how much olfactory cues (e.g. pheromones) actually could have been avoided as intended.
The average walking velocity over all experiments was $1.73 cm/s$.
\\
\\
Acquiring the velocity through the $MSD$ and relation $\langle r^{2}(\tau)\rangle = (v \tau)^{2} + 2nD\tau$ is a viable way as well \cite{Cherstvy2021} \cite{Tseng2002} but leads to velocities that are on average $1cm/s$ lower, not reflecting reality (see figure \ref{fig:Lin_Poly_Exp_fit}).\\
What this relation however does offer, is to look at the translational diffusion coefficient $D_T$ and the general progression of the mean squared displacement $\langle r^{2}(\tau)\rangle$.
In chapter \ref{cha:Results} several different attempts to acquire $D_T$ were discussed:
In a first instance it appeared that the initial increase of the $MSD$ is linear (see figure \ref{fig:MSD_lin_fit}) for timescales $\tau < 20s$. Fitting a simple slope to this initial part of the displacement curve leads to a significant qualitative differentiation between the movement types.
Looking at the double logarithmic plot (figure \ref{fig:Lin_Poly_Exp_fit}) we see that especially for such short timescales, the $MSD$ never really is linear and that a power law would be a superior representation. Such a connection between the diffusion coefficient and a power law would need to be investigated within the scope of a time dependent diffusion with a generalized diffusion coefficient $K_{\alpha}$ and its associated anomaly parameter $\alpha$ (see equation \ref{eq:Time_Dep_Diff_Coeff}). However, the numerical values for $K_{\alpha}$ and $\alpha$ are already included - it remains yet unclear how such a time dependent diffusion would influence the Langevin equation.
The third described method - looking at the graphs intersect with the y-axis in the double logarithmic plot - is in fact equivalent to the first, but differing in only considering $\tau < 1s$. Again only speaking in qualitative terms this leads to values with a scaling factor of $1/2$ compared to $D_T$ with $\tau<20s$, and has in general a lower deviation, as there is only so much diffusion able to happen in such a short amount of time. The reduction by a factor $1/2$ additionally indicates that on average the bees act more diffusive on larger timescales.
\\
\\
After examining the power spectrum of the velocity, acquired simply through the Fourier transformed signal (see figure \ref{fig:PSD_velocity}) it is striking that there for one is a flattening at low $\omega$ present only in $GF_{1-4}$, indicating a strong reaction of the velocity to a ``boundary''. Additionally we see that the spectrum for $IB_{1-4}$ resembles the spectrum of white noise, especially for high frequencies. Contrary to that stands the high frequency spectrum of the other three types, which scales with a factor $1/\omega$ and therefore being similar to pink noise. There exist numerous explanations for pink noise and one of them is the presence of a two state parameter in the data. In our particular context such a parameter can be explained by/found in the switching between two velocities $v=0$ and $v=v_{bee}$. Defining this in turn as being equivalent to a random telegraph signal, it is possible to analytically formulate the spectrum of such a switching behaviour (see equation \ref{eq:PSD_RTS}) and look at its scaling.
The analytical solution for the spectrum of the RTS fits well enough considering the fact that we are still dealing with a highly noisy system additionally to scarce data points for different variables in the equation. A truly random behaviour, as considered in equation \ref{eq:PSD_velocity}, would show itself through a continuous scaling over $1/\omega^{2}$, the fact that the exponent of the fit stays below a value of $\alpha_{RTS}<2$ throughout all bees indeed confirms that the spectrum of the velocity is dependent on the bees behaviour regarding the RTS. Within the analytical solution the stopping duration $t_{s}$ is of crucial importance for the scaling and highly influenced by the local temperature and its with the stopping duration that we once again see a clearer picture in regards of the four different behaviours. The dependency of $t_{s}$ on $\Delta T$ is following a relation $\tau \cdot e^{-\beta \Delta T}$ (see figure \ref{fig:t_stopping_nonlin_16}). As the bees show no directed movement one could argue that they have no memory regarding the temperature, the concept of ``warmer (or colder) than before'' is not present, but only the concept of ``warm or not warm enough'' for when the bee is standing still.
%It is interesting to keep in mind that Levy flights have a comparable spectrum \cite{LiZhao2012} which again is nothing short of a two state system, where the step length is governed by a Levy distribution, alternating between rather common small steps and less common big ones. 
\\
\\
Lastly we want to incorporate the impact the temperature has on the general mean velocity. It was shown in figure \ref{fig:vel_vs_dT} that there indeed is an inhibiting effect - bees walk on average slower at higher temperatures. This indirectly proportional relation can straight away be plugged into the drift term $(1/\gamma) \cdot \nabla V$ of equation \ref{eq:Langevin_reduce}. 
This eventually completes the model which is indeed sufficient to describe different types of movement. However modeling such a trajectory and making it comparable with the present data sets requires at least implementation of collisions with the boundary and an answer to the question whether this induces any distinguished behaviour which ideally again differs, especially for random walking and wall following behaviour.
So far it has been an unreliable and difficult endeavour to differentiate between $RW$ and $WF$, owed to their similar nature.
\\
\\
In conclusion, it was verified that there indeed is a way to objectively classify movement of an insect, by comparing a well understood equation of motion to the insects thermotaxis, a behavior which directs its locomotion up or down a gradient of temperature. Even if the gradient seems to have little effect on the turning, the absolute temperature has a clear influence on the stopping duration.
The next steps would be to build an agent based model, simulating different trajectories and answer the question of how to introduce a boundary to the system including what happens when an agent reaches it.
Upcoming work will furthermore attempt to try and expand the Langevin equation by a social component - as the truly intriguing phenomena happen on a level above, on the level of a swarm - and simulate the swarm dynamics with an updated agent based model.

\section*{Acknowledgements}
This work was made possible by the \emph{Artificial Life Laboratory} and supported by the \emph{COLIBRI} initiative at the University of Graz.


%Furthermore, even when only looking at the experiments with the steepest gradients, there seems not to be a dominant type of movement (check this again!), although the steepest gradient is expected to deliver the clearest picture regarding uphill walks. The question that arises from this is by how much the bees are ultimately reacting solely to the gradient, and by how much other factors, that manifest themselves in/as (perceived) noise of the system/inherent noise in the bees, play a part. 

%The ignored terms cana be addressed

%\cite{LiZhao2012} The reason for 1/f noise can also be found in levy flights and to some extent taht is also what happens with the bees but contrayry to levy flights 


%Discussion of wide variety of gradients and uncertainties and that reducing the dataset further would be bad and about the noise in biological systems. Additionally its still data and for future experiments it would be of interest to maybe look at gradeint behaviour in a different range (24-28 and 35-39) whereas here we would be bound by physiological constraints where 10 degrees leads to kältestarre and 40 degrees already to the first Proteinstockung, whereas it is useful t oknow that bees have processes to minimize the damage on their organism, nevertheless this could be regarded as extreme stress that needs to be prevented. mathematically also singularity.

%The concept of a group can be introduced and friction can be estimated through including socail concept. How exaytly this would be adressed remains to be seen.

%A model will be developed to simulate brood dynamics and predict 



 