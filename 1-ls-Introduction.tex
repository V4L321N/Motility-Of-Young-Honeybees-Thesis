\chapter{Introduction}

Ethology, i.e. the study of animal behaviour is gaining traction in the last couple of decades, and through the technological advancements and wider availability of computing resources and algorithms that are getting constantly more efficient this field has now access to huge amounts of data.
Analysing this data and ultimately understanding and even predicting animal behaviour is of crucial importance for  health assessment, hazard avoidance or productivity enhancement.
Through the usage of pesticides, combustion of fossil fuels and the pollution of our environments in general, the human race has reached critical points in time where loss of biodiversity through another imminent mass extinction event in the world of insects may be part of our future \cite{Schachat2020} \cite{mengoni2018}.
\\
Within the insect realm exists an already well understood and in fact domesticated species: \textit{Apis mellifera}, the honeybee.
These small organisms already show a multitude of different individual behaviours, that lead to a variety of emerging traits of the collective. \cite{szopek2013} \cite{radspieler2009} \cite{schmickl2007}
\\
In a bee hive, where every individual bee has its age dependent role, with distinct behavioural patterns, it is known that freshly hatched bees are responsible for the cleaning of the brood nest and the brood cells, preparing them for new eggs to be laid \cite{baracchi2014}.
The brood nest is hereby actively heated to maintain a temperature of around 36\textdegree C, which is the required ambient temperature for the bees larvae to develop \cite{petz2004}. 
It has been shown that groups of freshly hatched honeybees which have been exposed to a temperature gradient predominantly position themselves at around 36\textdegree C, as they are still too young to produce heat on their own \cite{heran1952}.
\\
Quantifying this apparently simple thermotactic behaviour of an individual and encapsulating it into an analytical description in all its details is only possible to some extent, as biological systems are very often non-equilibrium in nature and affected by significant amounts of noise. An elegant way to describe such a system is through the theory of Brownian motion, developed by Scottish botanist Robert Brown \cite{Brown1827}, and later extended by Einstein \cite{Einstein1905}, Smoluchowski \cite{Smoluchowski1924} and Langevin \cite{Langevin1908}.
The theory of Brownian motion is already widely used and an affirmed tool for the motion analysis of not only inanimate objects like particles \cite{de2007} \cite{ruthardt2011} \cite{johnson2013} \cite{Mousavi2017} but also of comparatively more complex organic systems like nematodes \cite{ben2009}, flies \cite{ravbar2019} \cite{valente2007} \cite{connolly1967} or cognitively advanced mammals \cite{fonio2009} \cite{de2014}.
\\
The work presented here portrays an attempt to analyze the trajectories of single bees and describe them via the Langevin equation, a non-linear stochastic differential equation, and thereby open new possibilities to understand the dynamics of the whole colony through the actions of individuals. This description rests upon a solid and thorough physical groundwork and opens opportunities to use the entirety of statistical physics for the study of and insight in the microscopic as well as the macroscopic layers of the biological system ``bee''.
\\
This method aims to identify and classify different characteristics, that are presumably individually different for each bee and therefore define the performance of a swarm through their mixture.
\\
To achieve that, we will first test a variety of different assumptions in and approaches to the Langevin equation trying to identify a suitable physical model, that describes the average thermo- and ortho-kinesis of a bee.
We then will use the parameters of the model to distinguish between typical types of different behaviours, which have been defined beforehand and identified manually.
\\
First a short description of the conducted experiments including their setup and settings in chapter \ref{Cha:Experiments} is given, followed by the mathematical formulation of a system of customized differential equations and the needed means of analysis in chapter \ref{cha:Methods}. The extracted values and parameters that then define the bee movement through the Langevin equation are presented in chapter \ref{cha:Results}, closing with a conclusion and an outlook in chapter \ref{cha:Discussion}.

